\documentclass[11pt]{article}

    \usepackage[breakable]{tcolorbox}
    \usepackage{parskip} % Stop auto-indenting (to mimic markdown behaviour)
    
    \usepackage{iftex}
    \ifPDFTeX
    	\usepackage[T1]{fontenc}
    	\usepackage{mathpazo}
    \else
    	\usepackage{fontspec}
    \fi

    % Basic figure setup, for now with no caption control since it's done
    % automatically by Pandoc (which extracts ![](path) syntax from Markdown).
    \usepackage{graphicx}
    % Maintain compatibility with old templates. Remove in nbconvert 6.0
    \let\Oldincludegraphics\includegraphics
    % Ensure that by default, figures have no caption (until we provide a
    % proper Figure object with a Caption API and a way to capture that
    % in the conversion process - todo).
    \usepackage{caption}
    \DeclareCaptionFormat{nocaption}{}
    \captionsetup{format=nocaption,aboveskip=0pt,belowskip=0pt}

    \usepackage[Export]{adjustbox} % Used to constrain images to a maximum size
    \adjustboxset{max size={0.9\linewidth}{0.9\paperheight}}
    \usepackage{float}
    \floatplacement{figure}{H} % forces figures to be placed at the correct location
    \usepackage{xcolor} % Allow colors to be defined
    \usepackage{enumerate} % Needed for markdown enumerations to work
    \usepackage{geometry} % Used to adjust the document margins
    \usepackage{amsmath} % Equations
    \usepackage{amssymb} % Equations
    \usepackage{textcomp} % defines textquotesingle
    % Hack from http://tex.stackexchange.com/a/47451/13684:
    \AtBeginDocument{%
        \def\PYZsq{\textquotesingle}% Upright quotes in Pygmentized code
    }
    \usepackage{upquote} % Upright quotes for verbatim code
    \usepackage{eurosym} % defines \euro
    \usepackage[mathletters]{ucs} % Extended unicode (utf-8) support
    \usepackage{fancyvrb} % verbatim replacement that allows latex
    \usepackage{grffile} % extends the file name processing of package graphics 
                         % to support a larger range
    \makeatletter % fix for grffile with XeLaTeX
    \def\Gread@@xetex#1{%
      \IfFileExists{"\Gin@base".bb}%
      {\Gread@eps{\Gin@base.bb}}%
      {\Gread@@xetex@aux#1}%
    }
    \makeatother

    % The hyperref package gives us a pdf with properly built
    % internal navigation ('pdf bookmarks' for the table of contents,
    % internal cross-reference links, web links for URLs, etc.)
    \usepackage{hyperref}
    % The default LaTeX title has an obnoxious amount of whitespace. By default,
    % titling removes some of it. It also provides customization options.
    \usepackage{titling}
    \usepackage{longtable} % longtable support required by pandoc >1.10
    \usepackage{booktabs}  % table support for pandoc > 1.12.2
    \usepackage[inline]{enumitem} % IRkernel/repr support (it uses the enumerate* environment)
    \usepackage[normalem]{ulem} % ulem is needed to support strikethroughs (\sout)
                                % normalem makes italics be italics, not underlines
    \usepackage{mathrsfs}
    

    
    % Colors for the hyperref package
    \definecolor{urlcolor}{rgb}{0,.145,.698}
    \definecolor{linkcolor}{rgb}{.71,0.21,0.01}
    \definecolor{citecolor}{rgb}{.12,.54,.11}

    % ANSI colors
    \definecolor{ansi-black}{HTML}{3E424D}
    \definecolor{ansi-black-intense}{HTML}{282C36}
    \definecolor{ansi-red}{HTML}{E75C58}
    \definecolor{ansi-red-intense}{HTML}{B22B31}
    \definecolor{ansi-green}{HTML}{00A250}
    \definecolor{ansi-green-intense}{HTML}{007427}
    \definecolor{ansi-yellow}{HTML}{DDB62B}
    \definecolor{ansi-yellow-intense}{HTML}{B27D12}
    \definecolor{ansi-blue}{HTML}{208FFB}
    \definecolor{ansi-blue-intense}{HTML}{0065CA}
    \definecolor{ansi-magenta}{HTML}{D160C4}
    \definecolor{ansi-magenta-intense}{HTML}{A03196}
    \definecolor{ansi-cyan}{HTML}{60C6C8}
    \definecolor{ansi-cyan-intense}{HTML}{258F8F}
    \definecolor{ansi-white}{HTML}{C5C1B4}
    \definecolor{ansi-white-intense}{HTML}{A1A6B2}
    \definecolor{ansi-default-inverse-fg}{HTML}{FFFFFF}
    \definecolor{ansi-default-inverse-bg}{HTML}{000000}

    % commands and environments needed by pandoc snippets
    % extracted from the output of `pandoc -s`
    \providecommand{\tightlist}{%
      \setlength{\itemsep}{0pt}\setlength{\parskip}{0pt}}
    \DefineVerbatimEnvironment{Highlighting}{Verbatim}{commandchars=\\\{\}}
    % Add ',fontsize=\small' for more characters per line
    \newenvironment{Shaded}{}{}
    \newcommand{\KeywordTok}[1]{\textcolor[rgb]{0.00,0.44,0.13}{\textbf{{#1}}}}
    \newcommand{\DataTypeTok}[1]{\textcolor[rgb]{0.56,0.13,0.00}{{#1}}}
    \newcommand{\DecValTok}[1]{\textcolor[rgb]{0.25,0.63,0.44}{{#1}}}
    \newcommand{\BaseNTok}[1]{\textcolor[rgb]{0.25,0.63,0.44}{{#1}}}
    \newcommand{\FloatTok}[1]{\textcolor[rgb]{0.25,0.63,0.44}{{#1}}}
    \newcommand{\CharTok}[1]{\textcolor[rgb]{0.25,0.44,0.63}{{#1}}}
    \newcommand{\StringTok}[1]{\textcolor[rgb]{0.25,0.44,0.63}{{#1}}}
    \newcommand{\CommentTok}[1]{\textcolor[rgb]{0.38,0.63,0.69}{\textit{{#1}}}}
    \newcommand{\OtherTok}[1]{\textcolor[rgb]{0.00,0.44,0.13}{{#1}}}
    \newcommand{\AlertTok}[1]{\textcolor[rgb]{1.00,0.00,0.00}{\textbf{{#1}}}}
    \newcommand{\FunctionTok}[1]{\textcolor[rgb]{0.02,0.16,0.49}{{#1}}}
    \newcommand{\RegionMarkerTok}[1]{{#1}}
    \newcommand{\ErrorTok}[1]{\textcolor[rgb]{1.00,0.00,0.00}{\textbf{{#1}}}}
    \newcommand{\NormalTok}[1]{{#1}}
    
    % Additional commands for more recent versions of Pandoc
    \newcommand{\ConstantTok}[1]{\textcolor[rgb]{0.53,0.00,0.00}{{#1}}}
    \newcommand{\SpecialCharTok}[1]{\textcolor[rgb]{0.25,0.44,0.63}{{#1}}}
    \newcommand{\VerbatimStringTok}[1]{\textcolor[rgb]{0.25,0.44,0.63}{{#1}}}
    \newcommand{\SpecialStringTok}[1]{\textcolor[rgb]{0.73,0.40,0.53}{{#1}}}
    \newcommand{\ImportTok}[1]{{#1}}
    \newcommand{\DocumentationTok}[1]{\textcolor[rgb]{0.73,0.13,0.13}{\textit{{#1}}}}
    \newcommand{\AnnotationTok}[1]{\textcolor[rgb]{0.38,0.63,0.69}{\textbf{\textit{{#1}}}}}
    \newcommand{\CommentVarTok}[1]{\textcolor[rgb]{0.38,0.63,0.69}{\textbf{\textit{{#1}}}}}
    \newcommand{\VariableTok}[1]{\textcolor[rgb]{0.10,0.09,0.49}{{#1}}}
    \newcommand{\ControlFlowTok}[1]{\textcolor[rgb]{0.00,0.44,0.13}{\textbf{{#1}}}}
    \newcommand{\OperatorTok}[1]{\textcolor[rgb]{0.40,0.40,0.40}{{#1}}}
    \newcommand{\BuiltInTok}[1]{{#1}}
    \newcommand{\ExtensionTok}[1]{{#1}}
    \newcommand{\PreprocessorTok}[1]{\textcolor[rgb]{0.74,0.48,0.00}{{#1}}}
    \newcommand{\AttributeTok}[1]{\textcolor[rgb]{0.49,0.56,0.16}{{#1}}}
    \newcommand{\InformationTok}[1]{\textcolor[rgb]{0.38,0.63,0.69}{\textbf{\textit{{#1}}}}}
    \newcommand{\WarningTok}[1]{\textcolor[rgb]{0.38,0.63,0.69}{\textbf{\textit{{#1}}}}}
    
    
    % Define a nice break command that doesn't care if a line doesn't already
    % exist.
    \def\br{\hspace*{\fill} \\* }
    % Math Jax compatibility definitions
    \def\gt{>}
    \def\lt{<}
    \let\Oldtex\TeX
    \let\Oldlatex\LaTeX
    \renewcommand{\TeX}{\textrm{\Oldtex}}
    \renewcommand{\LaTeX}{\textrm{\Oldlatex}}
    % Document parameters
    % Document title
    \title{EE2703: Assignment 6}
    \author{Yogesh Agarwala \\ EE19B130}
    \date{April 11, 2021}
    
    
    
    
    
% Pygments definitions
\makeatletter
\def\PY@reset{\let\PY@it=\relax \let\PY@bf=\relax%
    \let\PY@ul=\relax \let\PY@tc=\relax%
    \let\PY@bc=\relax \let\PY@ff=\relax}
\def\PY@tok#1{\csname PY@tok@#1\endcsname}
\def\PY@toks#1+{\ifx\relax#1\empty\else%
    \PY@tok{#1}\expandafter\PY@toks\fi}
\def\PY@do#1{\PY@bc{\PY@tc{\PY@ul{%
    \PY@it{\PY@bf{\PY@ff{#1}}}}}}}
\def\PY#1#2{\PY@reset\PY@toks#1+\relax+\PY@do{#2}}

\expandafter\def\csname PY@tok@w\endcsname{\def\PY@tc##1{\textcolor[rgb]{0.73,0.73,0.73}{##1}}}
\expandafter\def\csname PY@tok@c\endcsname{\let\PY@it=\textit\def\PY@tc##1{\textcolor[rgb]{0.25,0.50,0.50}{##1}}}
\expandafter\def\csname PY@tok@cp\endcsname{\def\PY@tc##1{\textcolor[rgb]{0.74,0.48,0.00}{##1}}}
\expandafter\def\csname PY@tok@k\endcsname{\let\PY@bf=\textbf\def\PY@tc##1{\textcolor[rgb]{0.00,0.50,0.00}{##1}}}
\expandafter\def\csname PY@tok@kp\endcsname{\def\PY@tc##1{\textcolor[rgb]{0.00,0.50,0.00}{##1}}}
\expandafter\def\csname PY@tok@kt\endcsname{\def\PY@tc##1{\textcolor[rgb]{0.69,0.00,0.25}{##1}}}
\expandafter\def\csname PY@tok@o\endcsname{\def\PY@tc##1{\textcolor[rgb]{0.40,0.40,0.40}{##1}}}
\expandafter\def\csname PY@tok@ow\endcsname{\let\PY@bf=\textbf\def\PY@tc##1{\textcolor[rgb]{0.67,0.13,1.00}{##1}}}
\expandafter\def\csname PY@tok@nb\endcsname{\def\PY@tc##1{\textcolor[rgb]{0.00,0.50,0.00}{##1}}}
\expandafter\def\csname PY@tok@nf\endcsname{\def\PY@tc##1{\textcolor[rgb]{0.00,0.00,1.00}{##1}}}
\expandafter\def\csname PY@tok@nc\endcsname{\let\PY@bf=\textbf\def\PY@tc##1{\textcolor[rgb]{0.00,0.00,1.00}{##1}}}
\expandafter\def\csname PY@tok@nn\endcsname{\let\PY@bf=\textbf\def\PY@tc##1{\textcolor[rgb]{0.00,0.00,1.00}{##1}}}
\expandafter\def\csname PY@tok@ne\endcsname{\let\PY@bf=\textbf\def\PY@tc##1{\textcolor[rgb]{0.82,0.25,0.23}{##1}}}
\expandafter\def\csname PY@tok@nv\endcsname{\def\PY@tc##1{\textcolor[rgb]{0.10,0.09,0.49}{##1}}}
\expandafter\def\csname PY@tok@no\endcsname{\def\PY@tc##1{\textcolor[rgb]{0.53,0.00,0.00}{##1}}}
\expandafter\def\csname PY@tok@nl\endcsname{\def\PY@tc##1{\textcolor[rgb]{0.63,0.63,0.00}{##1}}}
\expandafter\def\csname PY@tok@ni\endcsname{\let\PY@bf=\textbf\def\PY@tc##1{\textcolor[rgb]{0.60,0.60,0.60}{##1}}}
\expandafter\def\csname PY@tok@na\endcsname{\def\PY@tc##1{\textcolor[rgb]{0.49,0.56,0.16}{##1}}}
\expandafter\def\csname PY@tok@nt\endcsname{\let\PY@bf=\textbf\def\PY@tc##1{\textcolor[rgb]{0.00,0.50,0.00}{##1}}}
\expandafter\def\csname PY@tok@nd\endcsname{\def\PY@tc##1{\textcolor[rgb]{0.67,0.13,1.00}{##1}}}
\expandafter\def\csname PY@tok@s\endcsname{\def\PY@tc##1{\textcolor[rgb]{0.73,0.13,0.13}{##1}}}
\expandafter\def\csname PY@tok@sd\endcsname{\let\PY@it=\textit\def\PY@tc##1{\textcolor[rgb]{0.73,0.13,0.13}{##1}}}
\expandafter\def\csname PY@tok@si\endcsname{\let\PY@bf=\textbf\def\PY@tc##1{\textcolor[rgb]{0.73,0.40,0.53}{##1}}}
\expandafter\def\csname PY@tok@se\endcsname{\let\PY@bf=\textbf\def\PY@tc##1{\textcolor[rgb]{0.73,0.40,0.13}{##1}}}
\expandafter\def\csname PY@tok@sr\endcsname{\def\PY@tc##1{\textcolor[rgb]{0.73,0.40,0.53}{##1}}}
\expandafter\def\csname PY@tok@ss\endcsname{\def\PY@tc##1{\textcolor[rgb]{0.10,0.09,0.49}{##1}}}
\expandafter\def\csname PY@tok@sx\endcsname{\def\PY@tc##1{\textcolor[rgb]{0.00,0.50,0.00}{##1}}}
\expandafter\def\csname PY@tok@m\endcsname{\def\PY@tc##1{\textcolor[rgb]{0.40,0.40,0.40}{##1}}}
\expandafter\def\csname PY@tok@gh\endcsname{\let\PY@bf=\textbf\def\PY@tc##1{\textcolor[rgb]{0.00,0.00,0.50}{##1}}}
\expandafter\def\csname PY@tok@gu\endcsname{\let\PY@bf=\textbf\def\PY@tc##1{\textcolor[rgb]{0.50,0.00,0.50}{##1}}}
\expandafter\def\csname PY@tok@gd\endcsname{\def\PY@tc##1{\textcolor[rgb]{0.63,0.00,0.00}{##1}}}
\expandafter\def\csname PY@tok@gi\endcsname{\def\PY@tc##1{\textcolor[rgb]{0.00,0.63,0.00}{##1}}}
\expandafter\def\csname PY@tok@gr\endcsname{\def\PY@tc##1{\textcolor[rgb]{1.00,0.00,0.00}{##1}}}
\expandafter\def\csname PY@tok@ge\endcsname{\let\PY@it=\textit}
\expandafter\def\csname PY@tok@gs\endcsname{\let\PY@bf=\textbf}
\expandafter\def\csname PY@tok@gp\endcsname{\let\PY@bf=\textbf\def\PY@tc##1{\textcolor[rgb]{0.00,0.00,0.50}{##1}}}
\expandafter\def\csname PY@tok@go\endcsname{\def\PY@tc##1{\textcolor[rgb]{0.53,0.53,0.53}{##1}}}
\expandafter\def\csname PY@tok@gt\endcsname{\def\PY@tc##1{\textcolor[rgb]{0.00,0.27,0.87}{##1}}}
\expandafter\def\csname PY@tok@err\endcsname{\def\PY@bc##1{\setlength{\fboxsep}{0pt}\fcolorbox[rgb]{1.00,0.00,0.00}{1,1,1}{\strut ##1}}}
\expandafter\def\csname PY@tok@kc\endcsname{\let\PY@bf=\textbf\def\PY@tc##1{\textcolor[rgb]{0.00,0.50,0.00}{##1}}}
\expandafter\def\csname PY@tok@kd\endcsname{\let\PY@bf=\textbf\def\PY@tc##1{\textcolor[rgb]{0.00,0.50,0.00}{##1}}}
\expandafter\def\csname PY@tok@kn\endcsname{\let\PY@bf=\textbf\def\PY@tc##1{\textcolor[rgb]{0.00,0.50,0.00}{##1}}}
\expandafter\def\csname PY@tok@kr\endcsname{\let\PY@bf=\textbf\def\PY@tc##1{\textcolor[rgb]{0.00,0.50,0.00}{##1}}}
\expandafter\def\csname PY@tok@bp\endcsname{\def\PY@tc##1{\textcolor[rgb]{0.00,0.50,0.00}{##1}}}
\expandafter\def\csname PY@tok@fm\endcsname{\def\PY@tc##1{\textcolor[rgb]{0.00,0.00,1.00}{##1}}}
\expandafter\def\csname PY@tok@vc\endcsname{\def\PY@tc##1{\textcolor[rgb]{0.10,0.09,0.49}{##1}}}
\expandafter\def\csname PY@tok@vg\endcsname{\def\PY@tc##1{\textcolor[rgb]{0.10,0.09,0.49}{##1}}}
\expandafter\def\csname PY@tok@vi\endcsname{\def\PY@tc##1{\textcolor[rgb]{0.10,0.09,0.49}{##1}}}
\expandafter\def\csname PY@tok@vm\endcsname{\def\PY@tc##1{\textcolor[rgb]{0.10,0.09,0.49}{##1}}}
\expandafter\def\csname PY@tok@sa\endcsname{\def\PY@tc##1{\textcolor[rgb]{0.73,0.13,0.13}{##1}}}
\expandafter\def\csname PY@tok@sb\endcsname{\def\PY@tc##1{\textcolor[rgb]{0.73,0.13,0.13}{##1}}}
\expandafter\def\csname PY@tok@sc\endcsname{\def\PY@tc##1{\textcolor[rgb]{0.73,0.13,0.13}{##1}}}
\expandafter\def\csname PY@tok@dl\endcsname{\def\PY@tc##1{\textcolor[rgb]{0.73,0.13,0.13}{##1}}}
\expandafter\def\csname PY@tok@s2\endcsname{\def\PY@tc##1{\textcolor[rgb]{0.73,0.13,0.13}{##1}}}
\expandafter\def\csname PY@tok@sh\endcsname{\def\PY@tc##1{\textcolor[rgb]{0.73,0.13,0.13}{##1}}}
\expandafter\def\csname PY@tok@s1\endcsname{\def\PY@tc##1{\textcolor[rgb]{0.73,0.13,0.13}{##1}}}
\expandafter\def\csname PY@tok@mb\endcsname{\def\PY@tc##1{\textcolor[rgb]{0.40,0.40,0.40}{##1}}}
\expandafter\def\csname PY@tok@mf\endcsname{\def\PY@tc##1{\textcolor[rgb]{0.40,0.40,0.40}{##1}}}
\expandafter\def\csname PY@tok@mh\endcsname{\def\PY@tc##1{\textcolor[rgb]{0.40,0.40,0.40}{##1}}}
\expandafter\def\csname PY@tok@mi\endcsname{\def\PY@tc##1{\textcolor[rgb]{0.40,0.40,0.40}{##1}}}
\expandafter\def\csname PY@tok@il\endcsname{\def\PY@tc##1{\textcolor[rgb]{0.40,0.40,0.40}{##1}}}
\expandafter\def\csname PY@tok@mo\endcsname{\def\PY@tc##1{\textcolor[rgb]{0.40,0.40,0.40}{##1}}}
\expandafter\def\csname PY@tok@ch\endcsname{\let\PY@it=\textit\def\PY@tc##1{\textcolor[rgb]{0.25,0.50,0.50}{##1}}}
\expandafter\def\csname PY@tok@cm\endcsname{\let\PY@it=\textit\def\PY@tc##1{\textcolor[rgb]{0.25,0.50,0.50}{##1}}}
\expandafter\def\csname PY@tok@cpf\endcsname{\let\PY@it=\textit\def\PY@tc##1{\textcolor[rgb]{0.25,0.50,0.50}{##1}}}
\expandafter\def\csname PY@tok@c1\endcsname{\let\PY@it=\textit\def\PY@tc##1{\textcolor[rgb]{0.25,0.50,0.50}{##1}}}
\expandafter\def\csname PY@tok@cs\endcsname{\let\PY@it=\textit\def\PY@tc##1{\textcolor[rgb]{0.25,0.50,0.50}{##1}}}

\def\PYZbs{\char`\\}
\def\PYZus{\char`\_}
\def\PYZob{\char`\{}
\def\PYZcb{\char`\}}
\def\PYZca{\char`\^}
\def\PYZam{\char`\&}
\def\PYZlt{\char`\<}
\def\PYZgt{\char`\>}
\def\PYZsh{\char`\#}
\def\PYZpc{\char`\%}
\def\PYZdl{\char`\$}
\def\PYZhy{\char`\-}
\def\PYZsq{\char`\'}
\def\PYZdq{\char`\"}
\def\PYZti{\char`\~}
% for compatibility with earlier versions
\def\PYZat{@}
\def\PYZlb{[}
\def\PYZrb{]}
\makeatother


    % For linebreaks inside Verbatim environment from package fancyvrb. 
    \makeatletter
        \newbox\Wrappedcontinuationbox 
        \newbox\Wrappedvisiblespacebox 
        \newcommand*\Wrappedvisiblespace {\textcolor{red}{\textvisiblespace}} 
        \newcommand*\Wrappedcontinuationsymbol {\textcolor{red}{\llap{\tiny$\m@th\hookrightarrow$}}} 
        \newcommand*\Wrappedcontinuationindent {3ex } 
        \newcommand*\Wrappedafterbreak {\kern\Wrappedcontinuationindent\copy\Wrappedcontinuationbox} 
        % Take advantage of the already applied Pygments mark-up to insert 
        % potential linebreaks for TeX processing. 
        %        {, <, #, %, $, ' and ": go to next line. 
        %        _, }, ^, &, >, - and ~: stay at end of broken line. 
        % Use of \textquotesingle for straight quote. 
        \newcommand*\Wrappedbreaksatspecials {% 
            \def\PYGZus{\discretionary{\char`\_}{\Wrappedafterbreak}{\char`\_}}% 
            \def\PYGZob{\discretionary{}{\Wrappedafterbreak\char`\{}{\char`\{}}% 
            \def\PYGZcb{\discretionary{\char`\}}{\Wrappedafterbreak}{\char`\}}}% 
            \def\PYGZca{\discretionary{\char`\^}{\Wrappedafterbreak}{\char`\^}}% 
            \def\PYGZam{\discretionary{\char`\&}{\Wrappedafterbreak}{\char`\&}}% 
            \def\PYGZlt{\discretionary{}{\Wrappedafterbreak\char`\<}{\char`\<}}% 
            \def\PYGZgt{\discretionary{\char`\>}{\Wrappedafterbreak}{\char`\>}}% 
            \def\PYGZsh{\discretionary{}{\Wrappedafterbreak\char`\#}{\char`\#}}% 
            \def\PYGZpc{\discretionary{}{\Wrappedafterbreak\char`\%}{\char`\%}}% 
            \def\PYGZdl{\discretionary{}{\Wrappedafterbreak\char`\$}{\char`\$}}% 
            \def\PYGZhy{\discretionary{\char`\-}{\Wrappedafterbreak}{\char`\-}}% 
            \def\PYGZsq{\discretionary{}{\Wrappedafterbreak\textquotesingle}{\textquotesingle}}% 
            \def\PYGZdq{\discretionary{}{\Wrappedafterbreak\char`\"}{\char`\"}}% 
            \def\PYGZti{\discretionary{\char`\~}{\Wrappedafterbreak}{\char`\~}}% 
        } 
        % Some characters . , ; ? ! / are not pygmentized. 
        % This macro makes them "active" and they will insert potential linebreaks 
        \newcommand*\Wrappedbreaksatpunct {% 
            \lccode`\~`\.\lowercase{\def~}{\discretionary{\hbox{\char`\.}}{\Wrappedafterbreak}{\hbox{\char`\.}}}% 
            \lccode`\~`\,\lowercase{\def~}{\discretionary{\hbox{\char`\,}}{\Wrappedafterbreak}{\hbox{\char`\,}}}% 
            \lccode`\~`\;\lowercase{\def~}{\discretionary{\hbox{\char`\;}}{\Wrappedafterbreak}{\hbox{\char`\;}}}% 
            \lccode`\~`\:\lowercase{\def~}{\discretionary{\hbox{\char`\:}}{\Wrappedafterbreak}{\hbox{\char`\:}}}% 
            \lccode`\~`\?\lowercase{\def~}{\discretionary{\hbox{\char`\?}}{\Wrappedafterbreak}{\hbox{\char`\?}}}% 
            \lccode`\~`\!\lowercase{\def~}{\discretionary{\hbox{\char`\!}}{\Wrappedafterbreak}{\hbox{\char`\!}}}% 
            \lccode`\~`\/\lowercase{\def~}{\discretionary{\hbox{\char`\/}}{\Wrappedafterbreak}{\hbox{\char`\/}}}% 
            \catcode`\.\active
            \catcode`\,\active 
            \catcode`\;\active
            \catcode`\:\active
            \catcode`\?\active
            \catcode`\!\active
            \catcode`\/\active 
            \lccode`\~`\~ 	
        }
    \makeatother

    \let\OriginalVerbatim=\Verbatim
    \makeatletter
    \renewcommand{\Verbatim}[1][1]{%
        %\parskip\z@skip
        \sbox\Wrappedcontinuationbox {\Wrappedcontinuationsymbol}%
        \sbox\Wrappedvisiblespacebox {\FV@SetupFont\Wrappedvisiblespace}%
        \def\FancyVerbFormatLine ##1{\hsize\linewidth
            \vtop{\raggedright\hyphenpenalty\z@\exhyphenpenalty\z@
                \doublehyphendemerits\z@\finalhyphendemerits\z@
                \strut ##1\strut}%
        }%
        % If the linebreak is at a space, the latter will be displayed as visible
        % space at end of first line, and a continuation symbol starts next line.
        % Stretch/shrink are however usually zero for typewriter font.
        \def\FV@Space {%
            \nobreak\hskip\z@ plus\fontdimen3\font minus\fontdimen4\font
            \discretionary{\copy\Wrappedvisiblespacebox}{\Wrappedafterbreak}
            {\kern\fontdimen2\font}%
        }%
        
        % Allow breaks at special characters using \PYG... macros.
        \Wrappedbreaksatspecials
        % Breaks at punctuation characters . , ; ? ! and / need catcode=\active 	
        \OriginalVerbatim[#1,codes*=\Wrappedbreaksatpunct]%
    }
    \makeatother

    % Exact colors from NB
    \definecolor{incolor}{HTML}{303F9F}
    \definecolor{outcolor}{HTML}{D84315}
    \definecolor{cellborder}{HTML}{CFCFCF}
    \definecolor{cellbackground}{HTML}{F7F7F7}
    
    % prompt
    \makeatletter
    \newcommand{\boxspacing}{\kern\kvtcb@left@rule\kern\kvtcb@boxsep}
    \makeatother
    \newcommand{\prompt}[4]{
        \ttfamily\llap{{\color{#2}[#3]:\hspace{3pt}#4}}\vspace{-\baselineskip}
    }
    

    
    % Prevent overflowing lines due to hard-to-break entities
    \sloppy 
    % Setup hyperref package
    \hypersetup{
      breaklinks=true,  % so long urls are correctly broken across lines
      colorlinks=true,
      urlcolor=urlcolor,
      linkcolor=linkcolor,
      citecolor=citecolor,
      }
    % Slightly bigger margins than the latex defaults
    
    \geometry{verbose,tmargin=1in,bmargin=1in,lmargin=1in,rmargin=1in}
    
    

\begin{document}
    
    \maketitle
    


    \hypertarget{introduction}{%
\section{Introduction}\label{introduction}}

In this assignment, we model a tubelight as a one dimensional space of
gas in which electrons are continually injected at the cathode and
accelerated towards the anode by a constant electric field. The
electrons can ionize material atoms if they achieve a velocity greater
than some threshold, leading to an emission of a photon. This ionization
is modeled as a random process. The tubelight is simulated for a certain
number of timesteps from an initial state of having no electrons. The
results obtained are plotted and studied.

    \begin{tcolorbox}[breakable, size=fbox, boxrule=1pt, pad at break*=1mm,colback=cellbackground, colframe=cellborder]
\prompt{In}{incolor}{2}{\boxspacing}
\begin{Verbatim}[commandchars=\\\{\}]
\PY{k+kn}{import} \PY{n+nn}{numpy} \PY{k}{as} \PY{n+nn}{np}
\PY{k+kn}{import} \PY{n+nn}{pandas} \PY{k}{as} \PY{n+nn}{pd}
\PY{k+kn}{from} \PY{n+nn}{pylab} \PY{k+kn}{import} \PY{o}{*}
\PY{k+kn}{import} \PY{n+nn}{sys}
\end{Verbatim}
\end{tcolorbox}

    \begin{tcolorbox}[breakable, size=fbox, boxrule=1pt, pad at break*=1mm,colback=cellbackground, colframe=cellborder]
\prompt{In}{incolor}{18}{\boxspacing}
\begin{Verbatim}[commandchars=\\\{\}]
\PY{l+s+sd}{\PYZdq{}\PYZdq{}\PYZdq{}}
\PY{l+s+sd}{    Taking arguments from the user through commandline}
\PY{l+s+sd}{    and if the arguments are not provided then the code }
\PY{l+s+sd}{    will run on the default values}
\PY{l+s+sd}{    where,}
\PY{l+s+sd}{    n: integer length of tubelight}
\PY{l+s+sd}{    M: average number of electrons generated per timestep}
\PY{l+s+sd}{    nk: total number of timesteps to simulate}
\PY{l+s+sd}{    u0: threshold voltage for ionization}
\PY{l+s+sd}{    p: probability of ionization given an electron is faster than the threshold}
\PY{l+s+sd}{    Msig: stddev of number of electrons generated per timestep}
\PY{l+s+sd}{\PYZdq{}\PYZdq{}\PYZdq{}}
\PY{c+c1}{\PYZsh{}command line input}
\PY{k}{if}\PY{p}{(}\PY{n+nb}{len}\PY{p}{(}\PY{n}{sys}\PY{o}{.}\PY{n}{argv}\PY{p}{)}\PY{o}{==}\PY{l+m+mi}{7}\PY{p}{)}\PY{p}{:}
    \PY{n}{n}\PY{p}{,}\PY{n}{M}\PY{p}{,}\PY{n}{nk}\PY{p}{,}\PY{n}{u0}\PY{p}{,}\PY{n}{p}\PY{p}{,}\PY{n}{Msig} \PY{o}{=} \PY{p}{[}\PY{n+nb}{int}\PY{p}{(}\PY{n}{x}\PY{p}{)} \PY{k}{for} \PY{n}{x} \PY{o+ow}{in} \PY{n}{sys}\PY{o}{.}\PY{n}{argv}\PY{p}{[}\PY{l+m+mi}{1}\PY{p}{:}\PY{l+m+mi}{7}\PY{p}{]} \PY{p}{]}

\PY{c+c1}{\PYZsh{}default arguments}
\PY{k}{else}\PY{p}{:}
    \PY{n}{n}\PY{o}{=} \PY{l+m+mi}{100}
    \PY{n}{M}\PY{o}{=}\PY{l+m+mi}{5}
    \PY{n}{nk}\PY{o}{=}\PY{l+m+mi}{500}
    \PY{n}{u0}\PY{o}{=}\PY{l+m+mi}{7}
    \PY{n}{p}\PY{o}{=}\PY{l+m+mf}{0.5}
    \PY{n}{Msig}\PY{o}{=}\PY{l+m+mi}{1}
\end{Verbatim}
\end{tcolorbox}

    \hypertarget{simulation-function}{%
\section{Simulation Function}\label{simulation-function}}

A function to simulate the tubelight given certain parameters is written
below:

    \begin{tcolorbox}[breakable, size=fbox, boxrule=1pt, pad at break*=1mm,colback=cellbackground, colframe=cellborder]
\prompt{In}{incolor}{45}{\boxspacing}
\begin{Verbatim}[commandchars=\\\{\}]
\PY{l+s+sd}{\PYZdq{}\PYZdq{}\PYZdq{}}
\PY{l+s+sd}{    Simulate a tubelight and return the electron positions}
\PY{l+s+sd}{    and velocities, and positions of photon emissions.}
\PY{l+s+sd}{\PYZdq{}\PYZdq{}\PYZdq{}}
\PY{k}{def} \PY{n+nf}{simulateTubelight}\PY{p}{(}\PY{n}{n}\PY{p}{,}\PY{n}{M}\PY{p}{,}\PY{n}{nk}\PY{p}{,}\PY{n}{u0}\PY{p}{,}\PY{n}{p}\PY{p}{,}\PY{n}{Msig}\PY{p}{)}\PY{p}{:}

    \PY{n}{xx} \PY{o}{=} \PY{n}{zeros}\PY{p}{(}\PY{n}{n}\PY{o}{*}\PY{n}{M}\PY{p}{)}
    \PY{n}{u} \PY{o}{=} \PY{n}{zeros}\PY{p}{(}\PY{n}{n}\PY{o}{*}\PY{n}{M}\PY{p}{)}
    \PY{n}{dx} \PY{o}{=} \PY{n}{zeros}\PY{p}{(}\PY{n}{n}\PY{o}{*}\PY{n}{M}\PY{p}{)}

    \PY{n}{I} \PY{o}{=} \PY{p}{[}\PY{p}{]}
    \PY{n}{X} \PY{o}{=} \PY{p}{[}\PY{p}{]}
    \PY{n}{V} \PY{o}{=} \PY{p}{[}\PY{p}{]}

    \PY{k}{for} \PY{n}{k} \PY{o+ow}{in} \PY{n+nb}{range}\PY{p}{(}\PY{n}{nk}\PY{p}{)}\PY{p}{:}

        \PY{c+c1}{\PYZsh{} add new electrons}
        \PY{n}{m}\PY{o}{=}\PY{n+nb}{int}\PY{p}{(}\PY{n}{randn}\PY{p}{(}\PY{p}{)}\PY{o}{*}\PY{n}{Msig}\PY{o}{+}\PY{n}{M}\PY{p}{)}
        \PY{n}{jj} \PY{o}{=} \PY{n}{where}\PY{p}{(}\PY{n}{xx}\PY{o}{==}\PY{l+m+mi}{0}\PY{p}{)}
        \PY{n}{xx}\PY{p}{[}\PY{n}{jj}\PY{p}{[}\PY{l+m+mi}{0}\PY{p}{]}\PY{p}{[}\PY{p}{:}\PY{n}{m}\PY{p}{]}\PY{p}{]}\PY{o}{=}\PY{l+m+mi}{1}

        \PY{c+c1}{\PYZsh{} find electron indices}
        \PY{n}{ii} \PY{o}{=} \PY{n}{where}\PY{p}{(}\PY{n}{xx}\PY{o}{\PYZgt{}}\PY{l+m+mi}{0}\PY{p}{)}

        \PY{c+c1}{\PYZsh{} add to history lists}
        \PY{n}{X}\PY{o}{.}\PY{n}{extend}\PY{p}{(}\PY{n}{xx}\PY{p}{[}\PY{n}{ii}\PY{p}{]}\PY{o}{.}\PY{n}{tolist}\PY{p}{(}\PY{p}{)}\PY{p}{)}
        \PY{n}{V}\PY{o}{.}\PY{n}{extend}\PY{p}{(}\PY{n}{u}\PY{p}{[}\PY{n}{ii}\PY{p}{]}\PY{o}{.}\PY{n}{tolist}\PY{p}{(}\PY{p}{)}\PY{p}{)}

        \PY{c+c1}{\PYZsh{} update positions and speed}
        \PY{n}{dx}\PY{p}{[}\PY{n}{ii}\PY{p}{]} \PY{o}{=} \PY{n}{u}\PY{p}{[}\PY{n}{ii}\PY{p}{]}\PY{o}{+}\PY{l+m+mf}{0.5}
        \PY{n}{xx}\PY{p}{[}\PY{n}{ii}\PY{p}{]}\PY{o}{+}\PY{o}{=}\PY{n}{dx}\PY{p}{[}\PY{n}{ii}\PY{p}{]}
        \PY{n}{u}\PY{p}{[}\PY{n}{ii}\PY{p}{]}\PY{o}{+}\PY{o}{=}\PY{l+m+mi}{1}

        \PY{c+c1}{\PYZsh{} anode check}
        \PY{n}{kk} \PY{o}{=} \PY{n}{where}\PY{p}{(}\PY{n}{xx}\PY{o}{\PYZgt{}}\PY{o}{=}\PY{n}{n}\PY{p}{)}
        \PY{n}{xx}\PY{p}{[}\PY{n}{kk}\PY{p}{]}\PY{o}{=}\PY{l+m+mi}{0}
        \PY{n}{u}\PY{p}{[}\PY{n}{kk}\PY{p}{]}\PY{o}{=}\PY{l+m+mi}{0}

        \PY{c+c1}{\PYZsh{} ionization check}
        \PY{n}{kk} \PY{o}{=} \PY{n}{where}\PY{p}{(}\PY{n}{u}\PY{o}{\PYZgt{}}\PY{o}{=}\PY{n}{u0}\PY{p}{)}\PY{p}{[}\PY{l+m+mi}{0}\PY{p}{]}
        \PY{n}{ll}\PY{o}{=}\PY{n}{where}\PY{p}{(}\PY{n}{rand}\PY{p}{(}\PY{n+nb}{len}\PY{p}{(}\PY{n}{kk}\PY{p}{)}\PY{p}{)}\PY{o}{\PYZlt{}}\PY{o}{=}\PY{n}{p}\PY{p}{)}\PY{p}{;}
        \PY{n}{kl}\PY{o}{=}\PY{n}{kk}\PY{p}{[}\PY{n}{ll}\PY{p}{]}\PY{p}{;}

        \PY{c+c1}{\PYZsh{} ionize}
        \PY{n}{dt} \PY{o}{=} \PY{n}{rand}\PY{p}{(}\PY{n+nb}{len}\PY{p}{(}\PY{n}{kl}\PY{p}{)}\PY{p}{)}
        \PY{n}{xx}\PY{p}{[}\PY{n}{kl}\PY{p}{]}\PY{o}{=}\PY{n}{xx}\PY{p}{[}\PY{n}{kl}\PY{p}{]}\PY{o}{\PYZhy{}}\PY{n}{dx}\PY{p}{[}\PY{n}{kl}\PY{p}{]}\PY{o}{+}\PY{p}{(}\PY{p}{(}\PY{n}{u}\PY{p}{[}\PY{n}{kl}\PY{p}{]}\PY{o}{\PYZhy{}}\PY{l+m+mi}{1}\PY{p}{)}\PY{o}{*}\PY{n}{dt}\PY{o}{+}\PY{l+m+mf}{0.5}\PY{o}{*}\PY{n}{dt}\PY{o}{*}\PY{n}{dt}\PY{p}{)}
        \PY{n}{u}\PY{p}{[}\PY{n}{kl}\PY{p}{]}\PY{o}{=}\PY{l+m+mi}{0}

        \PY{c+c1}{\PYZsh{} add emissions}
        \PY{n}{I}\PY{o}{.}\PY{n}{extend}\PY{p}{(}\PY{n}{xx}\PY{p}{[}\PY{n}{kl}\PY{p}{]}\PY{o}{.}\PY{n}{tolist}\PY{p}{(}\PY{p}{)}\PY{p}{)}

    \PY{k}{return} \PY{n}{X}\PY{p}{,}\PY{n}{V}\PY{p}{,}\PY{n}{I}
\end{Verbatim}
\end{tcolorbox}

    \hypertarget{plots-function}{%
\section{Plots Function}\label{plots-function}}

A function to plot the required graphs is written below:

    \begin{tcolorbox}[breakable, size=fbox, boxrule=1pt, pad at break*=1mm,colback=cellbackground, colframe=cellborder]
\prompt{In}{incolor}{50}{\boxspacing}
\begin{Verbatim}[commandchars=\\\{\}]
\PY{l+s+sd}{\PYZdq{}\PYZdq{}\PYZdq{}}
\PY{l+s+sd}{    Plot histograms for X and I, and a phase space using X and V.}
\PY{l+s+sd}{    Returns the emission intensities and locations of histogram bins.}
\PY{l+s+sd}{\PYZdq{}\PYZdq{}\PYZdq{}}
\PY{k}{def} \PY{n+nf}{plotGraphs}\PY{p}{(}\PY{n}{X}\PY{p}{,}\PY{n}{V}\PY{p}{,}\PY{n}{I}\PY{p}{)}\PY{p}{:}
    
    \PY{c+c1}{\PYZsh{} electron density}
    \PY{n}{figure}\PY{p}{(}\PY{p}{)}
    \PY{n}{hist}\PY{p}{(}\PY{n}{X}\PY{p}{,}\PY{n}{bins}\PY{o}{=}\PY{n}{n}\PY{p}{,}\PY{n}{cumulative}\PY{o}{=}\PY{k+kc}{False}\PY{p}{)}
    \PY{n}{title}\PY{p}{(}\PY{l+s+s2}{\PYZdq{}}\PY{l+s+s2}{Electron density}\PY{l+s+s2}{\PYZdq{}}\PY{p}{)}
    \PY{n}{xlabel}\PY{p}{(}\PY{l+s+s2}{\PYZdq{}}\PY{l+s+s2}{\PYZdl{}x\PYZdl{}}\PY{l+s+s2}{\PYZdq{}}\PY{p}{)}
    \PY{n}{ylabel}\PY{p}{(}\PY{l+s+s2}{\PYZdq{}}\PY{l+s+s2}{Number of electrons}\PY{l+s+s2}{\PYZdq{}}\PY{p}{)}
    \PY{n}{show}\PY{p}{(}\PY{p}{)}

    \PY{c+c1}{\PYZsh{} emission instensity}
    \PY{n}{figure}\PY{p}{(}\PY{p}{)}
    \PY{n}{ints}\PY{p}{,}\PY{n}{bins}\PY{p}{,}\PY{n}{trash} \PY{o}{=} \PY{n}{hist}\PY{p}{(}\PY{n}{I}\PY{p}{,}\PY{n}{bins}\PY{o}{=}\PY{n}{n}\PY{p}{)}
    \PY{n}{title}\PY{p}{(}\PY{l+s+s2}{\PYZdq{}}\PY{l+s+s2}{Emission Intensity}\PY{l+s+s2}{\PYZdq{}}\PY{p}{)}
    \PY{n}{xlabel}\PY{p}{(}\PY{l+s+s2}{\PYZdq{}}\PY{l+s+s2}{\PYZdl{}x\PYZdl{}}\PY{l+s+s2}{\PYZdq{}}\PY{p}{)}
    \PY{n}{ylabel}\PY{p}{(}\PY{l+s+s2}{\PYZdq{}}\PY{l+s+s2}{I}\PY{l+s+s2}{\PYZdq{}}\PY{p}{)}
    \PY{n}{show}\PY{p}{(}\PY{p}{)}

    \PY{c+c1}{\PYZsh{} electron phase space}
    \PY{n}{figure}\PY{p}{(}\PY{p}{)}
    \PY{n}{scatter}\PY{p}{(}\PY{n}{X}\PY{p}{,}\PY{n}{V}\PY{p}{,}\PY{n}{marker}\PY{o}{=}\PY{l+s+s1}{\PYZsq{}}\PY{l+s+s1}{x}\PY{l+s+s1}{\PYZsq{}}\PY{p}{)}
    \PY{n}{title}\PY{p}{(}\PY{l+s+s2}{\PYZdq{}}\PY{l+s+s2}{Electron Phase Space}\PY{l+s+s2}{\PYZdq{}}\PY{p}{)}
    \PY{n}{xlabel}\PY{p}{(}\PY{l+s+s2}{\PYZdq{}}\PY{l+s+s2}{\PYZdl{}x\PYZdl{}}\PY{l+s+s2}{\PYZdq{}}\PY{p}{)}
    \PY{n}{ylabel}\PY{p}{(}\PY{l+s+s2}{\PYZdq{}}\PY{l+s+s2}{\PYZdl{}v\PYZdl{}}\PY{l+s+s2}{\PYZdq{}}\PY{p}{)}
    \PY{n}{show}\PY{p}{(}\PY{p}{)}
    
    \PY{k}{return} \PY{n}{ints}\PY{p}{,}\PY{n}{bins}
\end{Verbatim}
\end{tcolorbox}

    \hypertarget{running-the-simulation}{%
\section{Running the simulation}\label{running-the-simulation}}

The tubelight is simulated with the default parameters of n=100; M = 5;
nk =500; u0=7; p=0.5; Msig=0.1

    \begin{tcolorbox}[breakable, size=fbox, boxrule=1pt, pad at break*=1mm,colback=cellbackground, colframe=cellborder]
\prompt{In}{incolor}{61}{\boxspacing}
\begin{Verbatim}[commandchars=\\\{\}]
\PY{n}{n}\PY{o}{=}\PY{l+m+mi}{100}\PY{p}{;} \PY{n}{M} \PY{o}{=} \PY{l+m+mi}{5}\PY{p}{;} \PY{n}{nk} \PY{o}{=}\PY{l+m+mi}{500}\PY{p}{;} \PY{n}{u0}\PY{o}{=}\PY{l+m+mi}{7}\PY{p}{;} \PY{n}{p}\PY{o}{=}\PY{l+m+mf}{0.5}\PY{p}{;} \PY{n}{Msig}\PY{o}{=}\PY{l+m+mf}{0.1}
\end{Verbatim}
\end{tcolorbox}

    \begin{tcolorbox}[breakable, size=fbox, boxrule=1pt, pad at break*=1mm,colback=cellbackground, colframe=cellborder]
\prompt{In}{incolor}{62}{\boxspacing}
\begin{Verbatim}[commandchars=\\\{\}]
\PY{n}{X}\PY{p}{,}\PY{n}{V}\PY{p}{,}\PY{n}{I} \PY{o}{=} \PY{n}{simulateTubelight}\PY{p}{(}\PY{n}{n}\PY{p}{,}\PY{n}{M}\PY{p}{,}\PY{n}{nk}\PY{p}{,}\PY{n}{u0}\PY{p}{,}\PY{n}{p}\PY{p}{,}\PY{n}{Msig}\PY{p}{)}
\PY{n}{ints}\PY{p}{,} \PY{n}{bins} \PY{o}{=} \PY{n}{plotGraphs}\PY{p}{(}\PY{n}{X}\PY{p}{,}\PY{n}{V}\PY{p}{,}\PY{n}{I}\PY{p}{)}
\end{Verbatim}
\end{tcolorbox}

    \begin{center}
    \adjustimage{max size={0.9\linewidth}{0.9\paperheight}}{output_10_0.png}
    \end{center}
    { \hspace*{\fill} \\}
    
    \begin{center}
    \adjustimage{max size={0.9\linewidth}{0.9\paperheight}}{output_10_1.png}
    \end{center}
    { \hspace*{\fill} \\}
    
    \begin{center}
    \adjustimage{max size={0.9\linewidth}{0.9\paperheight}}{output_10_2.png}
    \end{center}
    { \hspace*{\fill} \\}
    
    \hypertarget{the-emission-count-for-each-value-of-x-is-tabulated-below}{%
\subsection{\texorpdfstring{The emission count for each value of \(x\)
is tabulated
below:}{The emission count for each value of x is tabulated below:}}\label{the-emission-count-for-each-value-of-x-is-tabulated-below}}

    \begin{tcolorbox}[breakable, size=fbox, boxrule=1pt, pad at break*=1mm,colback=cellbackground, colframe=cellborder]
\prompt{In}{incolor}{49}{\boxspacing}
\begin{Verbatim}[commandchars=\\\{\}]
\PY{n}{xpos}\PY{o}{=}\PY{l+m+mf}{0.5}\PY{o}{*}\PY{p}{(}\PY{n}{bins}\PY{p}{[}\PY{l+m+mi}{0}\PY{p}{:}\PY{o}{\PYZhy{}}\PY{l+m+mi}{1}\PY{p}{]}\PY{o}{+}\PY{n}{bins}\PY{p}{[}\PY{l+m+mi}{1}\PY{p}{:}\PY{p}{]}\PY{p}{)}
\PY{n}{M} \PY{o}{=} \PY{n}{np}\PY{o}{.}\PY{n}{c\PYZus{}}\PY{p}{[}\PY{n}{xpos}\PY{p}{,}\PY{n}{ints}\PY{p}{]}
\PY{n}{df} \PY{o}{=} \PY{n}{pd}\PY{o}{.}\PY{n}{DataFrame}\PY{p}{(}\PY{n}{M}\PY{p}{,}\PY{n}{columns}\PY{o}{=}\PY{p}{[}\PY{l+s+s1}{\PYZsq{}}\PY{l+s+s1}{xpos}\PY{l+s+s1}{\PYZsq{}}\PY{p}{,}\PY{l+s+s1}{\PYZsq{}}\PY{l+s+s1}{count}\PY{l+s+s1}{\PYZsq{}}\PY{p}{]}\PY{p}{)}
\PY{n+nb}{print}\PY{p}{(}\PY{l+s+s2}{\PYZdq{}}\PY{l+s+s2}{Intensity Data:}\PY{l+s+s2}{\PYZdq{}}\PY{p}{)}
\PY{n+nb}{print}\PY{p}{(}\PY{n}{df}\PY{o}{.}\PY{n}{to\PYZus{}string}\PY{p}{(}\PY{n}{index}\PY{o}{=}\PY{k+kc}{False}\PY{p}{)}\PY{p}{)}
\end{Verbatim}
\end{tcolorbox}

    \begin{Verbatim}[commandchars=\\\{\}]
Intensity Data:
      xpos  count
 19.405146  154.0
 20.209847  159.0
 21.014547  155.0
 21.819247  155.0
 22.623948  124.0
 23.428648  126.0
 24.233349  126.0
 25.038049  123.0
 25.842749   57.0
 26.647450   68.0
 27.452150   72.0
 28.256850   68.0
 29.061551   62.0
 29.866251   64.0
 30.670951   72.0
 31.475652   59.0
 32.280352   55.0
 33.085053   43.0
 33.889753   30.0
 34.694453   21.0
 35.499154   26.0
 36.303854   30.0
 37.108554   27.0
 37.913255   48.0
 38.717955   49.0
 39.522656   61.0
 40.327356   65.0
 41.132056   58.0
 41.936757   71.0
 42.741457   62.0
 43.546157   82.0
 44.350858   81.0
 45.155558   62.0
 45.960258   77.0
 46.764959   79.0
 47.569659   71.0
 48.374360   81.0
 49.179060   70.0
 49.983760   67.0
 50.788461   67.0
 51.593161   67.0
 52.397861   70.0
 53.202562   64.0
 54.007262   49.0
 54.811963   45.0
 55.616663   59.0
 56.421363   50.0
 57.226064   50.0
 58.030764   60.0
 58.835464   42.0
 59.640165   57.0
 60.444865   53.0
 61.249566   55.0
 62.054266   62.0
 62.858966   51.0
 63.663667   48.0
 64.468367   59.0
 65.273067   63.0
 66.077768   71.0
 66.882468   66.0
 67.687168   65.0
 68.491869   63.0
 69.296569   70.0
 70.101270   65.0
 70.905970   69.0
 71.710670   66.0
 72.515371   66.0
 73.320071   61.0
 74.124771   49.0
 74.929472   47.0
 75.734172   63.0
 76.538873   57.0
 77.343573   54.0
 78.148273   67.0
 78.952974   64.0
 79.757674   52.0
 80.562374   54.0
 81.367075   64.0
 82.171775   62.0
 82.976475   62.0
 83.781176   70.0
 84.585876   52.0
 85.390577   69.0
 86.195277   49.0
 86.999977   60.0
 87.804678   64.0
 88.609378   51.0
 89.414078   68.0
 90.218779   66.0
 91.023479   61.0
 91.828180   64.0
 92.632880   50.0
 93.437580   59.0
 94.242281   44.0
 95.046981   39.0
 95.851681   43.0
 96.656382   27.0
 97.461082   18.0
 98.265783    4.0
 99.070483    5.0
    \end{Verbatim}

    \hypertarget{altering-simulation-parameters}{%
\section{Altering Simulation
Parameters}\label{altering-simulation-parameters}}

We try out the following set of parameters: 1. n=100, M = 5, nk =1000,
u0=12, p=0.5,Msig=0.2 (larger threshold velocity). 2. n=100, M = 5, nk
=1000, u0=7, p=0.1 ,Msig=0.2 (lower probabiltiy of collision). 3. n=100,
M = 5, nk =1000, u0=7, p=0.1 ,Msig=4 (higher variance of
randomness(normal variable)).

    \hypertarget{larger-threshold-velocity}{%
\paragraph{Larger threshold velocity}\label{larger-threshold-velocity}}

    \begin{tcolorbox}[breakable, size=fbox, boxrule=1pt, pad at break*=1mm,colback=cellbackground, colframe=cellborder]
\prompt{In}{incolor}{59}{\boxspacing}
\begin{Verbatim}[commandchars=\\\{\}]
\PY{n}{n}\PY{o}{=}\PY{l+m+mi}{100}\PY{p}{;} \PY{n}{M} \PY{o}{=} \PY{l+m+mi}{5}\PY{p}{;} \PY{n}{nk} \PY{o}{=}\PY{l+m+mi}{1000}\PY{p}{;} \PY{n}{u0}\PY{o}{=}\PY{l+m+mi}{12}\PY{p}{;} \PY{n}{p}\PY{o}{=}\PY{l+m+mf}{0.5}\PY{p}{;} \PY{n}{Msig}\PY{o}{=}\PY{l+m+mf}{0.2}
\end{Verbatim}
\end{tcolorbox}

    \begin{tcolorbox}[breakable, size=fbox, boxrule=1pt, pad at break*=1mm,colback=cellbackground, colframe=cellborder]
\prompt{In}{incolor}{60}{\boxspacing}
\begin{Verbatim}[commandchars=\\\{\}]
\PY{n}{X}\PY{p}{,}\PY{n}{V}\PY{p}{,}\PY{n}{I} \PY{o}{=} \PY{n}{simulateTubelight}\PY{p}{(}\PY{n}{n}\PY{p}{,}\PY{n}{M}\PY{p}{,}\PY{n}{nk}\PY{p}{,}\PY{n}{u0}\PY{p}{,}\PY{n}{p}\PY{p}{,}\PY{n}{Msig}\PY{p}{)}
\PY{n}{ints}\PY{p}{,} \PY{n}{bins} \PY{o}{=} \PY{n}{plotGraphs}\PY{p}{(}\PY{n}{X}\PY{p}{,}\PY{n}{V}\PY{p}{,}\PY{n}{I}\PY{p}{)}
\end{Verbatim}
\end{tcolorbox}

    \begin{center}
    \adjustimage{max size={0.9\linewidth}{0.9\paperheight}}{output_16_0.png}
    \end{center}
    { \hspace*{\fill} \\}
    
    \begin{center}
    \adjustimage{max size={0.9\linewidth}{0.9\paperheight}}{output_16_1.png}
    \end{center}
    { \hspace*{\fill} \\}
    
    \begin{center}
    \adjustimage{max size={0.9\linewidth}{0.9\paperheight}}{output_16_2.png}
    \end{center}
    { \hspace*{\fill} \\}
    
    \hypertarget{lower-probabiltiy-of-collision}{%
\subsubsection{Lower probabiltiy of
collision}\label{lower-probabiltiy-of-collision}}

    \begin{tcolorbox}[breakable, size=fbox, boxrule=1pt, pad at break*=1mm,colback=cellbackground, colframe=cellborder]
\prompt{In}{incolor}{64}{\boxspacing}
\begin{Verbatim}[commandchars=\\\{\}]
\PY{n}{n}\PY{o}{=}\PY{l+m+mi}{100}\PY{p}{;} \PY{n}{M} \PY{o}{=} \PY{l+m+mi}{5}\PY{p}{;} \PY{n}{nk} \PY{o}{=}\PY{l+m+mi}{1000}\PY{p}{;} \PY{n}{u0}\PY{o}{=}\PY{l+m+mi}{7}\PY{p}{;} \PY{n}{p}\PY{o}{=}\PY{l+m+mf}{0.1}\PY{p}{;} \PY{n}{Msig}\PY{o}{=}\PY{l+m+mf}{0.2}
\end{Verbatim}
\end{tcolorbox}

    \begin{tcolorbox}[breakable, size=fbox, boxrule=1pt, pad at break*=1mm,colback=cellbackground, colframe=cellborder]
\prompt{In}{incolor}{65}{\boxspacing}
\begin{Verbatim}[commandchars=\\\{\}]
\PY{n}{X}\PY{p}{,}\PY{n}{V}\PY{p}{,}\PY{n}{I} \PY{o}{=} \PY{n}{simulateTubelight}\PY{p}{(}\PY{n}{n}\PY{p}{,}\PY{n}{M}\PY{p}{,}\PY{n}{nk}\PY{p}{,}\PY{n}{u0}\PY{p}{,}\PY{n}{p}\PY{p}{,}\PY{n}{Msig}\PY{p}{)}
\PY{n}{ints}\PY{p}{,} \PY{n}{bins} \PY{o}{=} \PY{n}{plotGraphs}\PY{p}{(}\PY{n}{X}\PY{p}{,}\PY{n}{V}\PY{p}{,}\PY{n}{I}\PY{p}{)}
\end{Verbatim}
\end{tcolorbox}

    \begin{center}
    \adjustimage{max size={0.9\linewidth}{0.9\paperheight}}{output_19_0.png}
    \end{center}
    { \hspace*{\fill} \\}
    
    \begin{center}
    \adjustimage{max size={0.9\linewidth}{0.9\paperheight}}{output_19_1.png}
    \end{center}
    { \hspace*{\fill} \\}
    
    \begin{center}
    \adjustimage{max size={0.9\linewidth}{0.9\paperheight}}{output_19_2.png}
    \end{center}
    { \hspace*{\fill} \\}
    
    \hypertarget{higher-variance-of-randomnessnormal-variable}{%
\subsubsection{Higher variance of randomness(normal
variable)}\label{higher-variance-of-randomnessnormal-variable}}

    \begin{tcolorbox}[breakable, size=fbox, boxrule=1pt, pad at break*=1mm,colback=cellbackground, colframe=cellborder]
\prompt{In}{incolor}{68}{\boxspacing}
\begin{Verbatim}[commandchars=\\\{\}]
\PY{n}{n}\PY{o}{=}\PY{l+m+mi}{100}\PY{p}{;} \PY{n}{M} \PY{o}{=} \PY{l+m+mi}{5}\PY{p}{;} \PY{n}{nk} \PY{o}{=}\PY{l+m+mi}{1000}\PY{p}{;} \PY{n}{u0}\PY{o}{=}\PY{l+m+mi}{7}\PY{p}{;} \PY{n}{p}\PY{o}{=}\PY{l+m+mf}{0.1}\PY{p}{;} \PY{n}{Msig}\PY{o}{=}\PY{l+m+mi}{4}
\end{Verbatim}
\end{tcolorbox}

    \begin{tcolorbox}[breakable, size=fbox, boxrule=1pt, pad at break*=1mm,colback=cellbackground, colframe=cellborder]
\prompt{In}{incolor}{70}{\boxspacing}
\begin{Verbatim}[commandchars=\\\{\}]
\PY{n}{X}\PY{p}{,}\PY{n}{V}\PY{p}{,}\PY{n}{I} \PY{o}{=} \PY{n}{simulateTubelight}\PY{p}{(}\PY{n}{n}\PY{p}{,}\PY{n}{M}\PY{p}{,}\PY{n}{nk}\PY{p}{,}\PY{n}{u0}\PY{p}{,}\PY{n}{p}\PY{p}{,}\PY{n}{Msig}\PY{p}{)}
\PY{n}{ints}\PY{p}{,} \PY{n}{bins} \PY{o}{=} \PY{n}{plotGraphs}\PY{p}{(}\PY{n}{X}\PY{p}{,}\PY{n}{V}\PY{p}{,}\PY{n}{I}\PY{p}{)}
\end{Verbatim}
\end{tcolorbox}

    \begin{center}
    \adjustimage{max size={0.9\linewidth}{0.9\paperheight}}{output_22_0.png}
    \end{center}
    { \hspace*{\fill} \\}
    
    \begin{center}
    \adjustimage{max size={0.9\linewidth}{0.9\paperheight}}{output_22_1.png}
    \end{center}
    { \hspace*{\fill} \\}
    
    \begin{center}
    \adjustimage{max size={0.9\linewidth}{0.9\paperheight}}{output_22_2.png}
    \end{center}
    { \hspace*{\fill} \\}
    
    \hypertarget{conclusion}{%
\section{Conclusion}\label{conclusion}}

This week's assignment covers using python to simulate models for
various requirements. In this case,we utilsise it for simualting
electron motion in a tubelight, and hence find out the illumination at
different points. The existence of an initial peak, and those of dark
patches. In the subsequent sections, we also went over the effect of
changing various parameters including probability of collision,
threshold velocities and standard deviation.

\hypertarget{we-can-make-the-following-observations-from-the-above-plots}{%
\paragraph{We can make the following observations from the above
plots:}\label{we-can-make-the-following-observations-from-the-above-plots}}

\begin{itemize}
\tightlist
\item
  The electron density is peaked at the initial parts of the tubelight
  as the electrons are gaining speed here and are not above the
  threshold. This means that the peaks are the positions of the
  electrons at the first few timesteps they experience. 
\item
  The peaks slowly smoothen out as \(x\) increases beyond \(19\). This
  is because the electrons achieve a threshold speed of \(7\) only after
  traversing a distance of \(19\) units. This means that they start
  ionizing the gas atoms and lose their speed due to an inelastic
  collision. 
\item
  The emission intensity also shows peaks which get diffused as \(x\)
  increases. This is due the same reason as above. Most of the electrons
  reach the threshold at roughly the same positions, leading to peaks in
  the number of photons emitted there. 
\item
  This phenomenon can also be seen in the phase space plot. Firstly, the
  velocities are restricted to discrete values, as the acceleration is
  set to \(1\), and we are not yet performing accurate velocity updates
  after collisions. 
\item
  One trajectory is separated from the rest of plot. This corresponds to
  those electrons which travel until the anode without suffering any
  inelastic collisions with gas atoms. This can be seen by noticing that
  the trajectory is parabolic. This means that
  \(v \space = k \sqrt{x}\), which is precisely the case for a particle
  moving with constant acceleration. 
\item
  The rest of the plot corresponds to the trajectories of those
  electrons which have suffered at least one collision with an atom.
  Since the collisions can occur over a continuous range of positions,
  the trajectories encompass all possible positions after \(x=19\). 
\item
  A gas which has a lower threshold velocity and a higher ionization
  probability is better suited for use in a tubelight, as it provides
  more uniform and a higher amount of photon emission intensity. 
\item
  The intensity histogram reveals that the electrons do not cause
  excitation of atoms till they cross a particular threshold velocity,
  as dictated by the nature of the gas used. Secondly, this gives rise
  to a peak in intensity just after the first mean length. This is
  beaucse a majority of electrons collide with atoms at this distance.
  Further this, subsequent peaks do exist, but have larger spread and
  are less prominent. We observer around 2 dark bands in this intensity
  profile. 
\item
  The electron phase plots show the constant acceleration all electrons
  initially undergo, and the subsequent random motion post collision.
  The phase plots are nearly uniformly distributed in the middle portion
  of the tubelight.
\end{itemize}


    % Add a bibliography block to the postdoc
    
    
    
\end{document}
